% 这是中国科学院大学计算机科学与技术专业《计算机组成原理(研讨课)》使用的实验报告 Latex 模板
% 本模板与 2024 年 2 月 Jun-xiong Ji 完成, 更改自由 Shing-Ho Lin 和 Jun-Xiong Ji 于 2022 年 9 月共同完成的基础物理实验模板
% 如有任何问题, 请联系: jijunxoing21@mails.ucas.ac.cn
% This is the LaTeX template for report of Experiment of Computer Organization and Design courses, based on its provided Word template. 
% This template is completed on Febrary 2024, based on the joint collabration of Shing-Ho Lin and Junxiong Ji in September 2022. 
% Adding numerous pictures and equations leads to unsatisfying experience in Word. Therefore LaTeX is better. 
% Feel free to contact me via: jijunxoing21@mails.ucas.ac.cn

\documentclass[11pt]{article}

\usepackage[a4paper]{geometry}
\geometry{left=2.0cm,right=2.0cm,top=2.5cm,bottom=2.5cm}

\usepackage{ctex} % 支持中文的LaTeX宏包
\usepackage{amsmath,amsfonts,graphicx,subfigure,amssymb,bm,amsthm,mathrsfs,mathtools,breqn} % 数学公式和符号的宏包集合
\usepackage{algorithm,algorithmicx} % 算法和伪代码
\usepackage[noend]{algpseudocode} % 算法和伪代码
\usepackage{fancyhdr} % 自定义页眉页脚
\usepackage[framemethod=TikZ]{mdframed} % 创建带边框的框架
\usepackage{fontspec} % 字体设置
\usepackage{adjustbox} % 调整盒子大小
\usepackage{fontsize} % 设置字体大小
\usepackage{tikz,xcolor} % 绘制图形和使用颜色
\usepackage{multicol} % 多栏排版
\usepackage{multirow} % 表格中合并单元格
\usepackage{pdfpages} % 插入PDF文件
\usepackage{listings} % 在文档中插入源代码
\usepackage{wrapfig} % 文字绕排图片
\usepackage{bigstrut,multirow,rotating} % 支持在表格中使用特殊命令
\usepackage{booktabs} % 创建美观的表格
\usepackage{circuitikz} % 绘制电路图
\usepackage{zhnumber} % 中文序号(用于标题)
\usepackage{tabularx} % 表格折行
\usepackage{float} % 限制图片浮动
\usetikzlibrary{positioning,arrows.meta}

\definecolor{dkgreen}{rgb}{0,0.6,0}
\definecolor{gray}{rgb}{0.5,0.5,0.5}
\definecolor{mauve}{rgb}{0.58,0,0.82}
\lstset{
  frame=tb,
  aboveskip=3mm,
  belowskip=3mm,
  showstringspaces=false,
  columns=flexible,
  framerule=1pt,
  rulecolor=\color{gray!35},
  backgroundcolor=\color{gray!5},
  basicstyle={\small\ttfamily},
  numbers=none,
  numberstyle=\tiny\color{gray},
  keywordstyle=\color{blue},
  commentstyle=\color{dkgreen},
  stringstyle=\color{mauve},
  breaklines=true,
  breakatwhitespace=true,
  tabsize=3,
}

% 轻松引用, 可以用\cref{}指令直接引用, 自动加前缀. 
% 例: 图片label为fig:1
% \cref{fig:1} => Figure.1
% \ref{fig:1}  => 1
\usepackage[capitalize]{cleveref}
% \crefname{section}{Sec.}{Secs.}
\Crefname{section}{Section}{Sections}
\Crefname{table}{Table}{Tables}
\crefname{table}{Table.}{Tabs.}

% \setmainfont{Palatino Linotype.ttf}
% \setCJKmainfont{SimHei.ttf}
% \setCJKsansfont{Songti.ttf}
% \setCJKmonofont{SimSun.ttf}
\punctstyle{kaiming}
% 偏好的几个字体, 可以根据需要自行加入字体ttf文件并调用

\renewcommand{\emph}[1]{\begin{kaishu}#1\end{kaishu}}

% 对 section 等环境的序号使用中文
\renewcommand \thesection{\zhnum{section}、}
\renewcommand \thesubsection{\arabic{subsection}}


%%%%%%%%%%%%%%%%%%%%%%%%%%%
%改这里可以修改实验报告表头的信息
\newcommand{\name}{许震宇}
\newcommand{\mydate}{2026.1.29}
% \newcommand{\major}{计算机科学与技术}

%%%%%%%%%%%%%%%%%%%%%%%%%%%

\begin{document}

\begin{center}
  \LARGE \bf UBS 实习汇报
\end{center}

\begin{center}
  \emph{汇报人} \underline{\makebox[7em][c]{\name}} 
  % 如果名字比较长, 可以修改box的长度"8em"为其他值
  \emph{日期} \underline{\makebox[12em][c]{\mydate}}
  % \emph{专业} \underline{\makebox[15em][c]{\major}}\\
\end{center}

  

\section{一个简短的自我介绍}

\begin{itemize}
  \item 许震宇
  \item 中国科学院大学(北京怀柔),分数线仅次于清北
  \item 家在深圳、港籍
  \item 徒步、网球、滑雪...
\end{itemize}

\noindent 
\textbf{微信:}
\begin{figure}[H]
  \centering
  \includegraphics[width=0.5\textwidth]{fig/qrcode.png}
  \caption{}
\end{figure}

\section{数据格式调研}

\begin{table}[H]
  \centering
  \renewcommand{\arraystretch}{1.2}
  \begin{tabularx}{\textwidth}{lXl}
    \toprule
    数据格式 & 适用场景与优点 & 备注 \\
    \midrule
    CSV & 通用、可读性强、便于快速查看与交换 & 体积大、读写慢、类型易丢失 \\
    Parquet & 列式存储、压缩好、读取快,适合大规模因子/行情特征 & 生态成熟(Spark/Arrow) \\
    Feather/Arrow & 序列化快、零拷贝传输,适合进程间共享与临时缓存 & 适合内存分析 \\
    HDF5 (.h5) & 分层结构、可存多表/多维数组、支持元数据与随机访问 & 适合大体量数值数据 \\
    SQLite & 轻量数据库、支持SQL查询,适合中小规模结构化数据 & 单文件便携 \\
    Pickle & Python对象持久化,开发效率高 & 跨语言差、版本兼容风险 \\
    JSON & 可读、跨语言、适合配置与元数据 & 体积大、速度慢 \\
    NPZ & 压缩数组打包,便于分发与存档 & 适合纯数值数组 \\
    \bottomrule
  \end{tabularx}
  \caption{量化常见数据格式对比}
\end{table}

\section{数据调研}
\begin{itemize}
  \item 数个指标从 2010 年到 2024 年的分时数据
  \item 层级:分时数据项 < 每日数据 < 从 2014 ~ 2024 的数据 < 3 个指标的从 2014 ~ 2024 的数据
  \item 每个分时数据项包含:Close     High      Low     Open  Time      Date        Amount     Volume
\end{itemize}







\end{document}
